\documentclass[12pt]{report}
\usepackage{amsthm}
\usepackage{amssymb}
\usepackage{amsmath}
\usepackage{listings}
\usepackage{graphicx}
\usepackage{float}
\usepackage{xcolor}
\usepackage{hyperref}

\hypersetup{
    colorlinks=true,
    linkcolor=black,
    filecolor=magenta,      
    urlcolor=blue,
    pdftitle={Overleaf Example},
    pdfpagemode=FullScreen,
}

\setlength{\parindent}{0pt}
\setlength{\parskip}{5pt}

\begin{document}
    \renewcommand{\labelenumii}{\arabic{enumi}.\arabic{enumii}}
    \renewcommand{\labelenumiii}{\arabic{enumi}.\arabic{enumii}.\arabic{enumiii}}
    \renewcommand{\labelenumiv}{\arabic{enumi}.\arabic{enumii}.\arabic{enumiii}.\arabic{enumiv}}

    \title{Progetto e-commerce}
    \author{A. Bandiera, I. Radu Barbalata, S. Bianco, S. Lazzaroni}
    \date{\today}

    \maketitle
    \tableofcontents
    \newpage

    \chapter{Introduzione}

    \section{Specifica dei requisiti}

    Si vuole progettare una piattaforma di e-commerce, nella quale grandi fornitori possono mettere in vendita prodotti per acquirenti privati.

    Un acquirente deve potersi registrare sulla piattaforma attraverso una e-mail, il proprio nome, il proprio cognome ed un recapito telefonico; inoltre, ogni acquirente deve indicare almeno un indirizzo per la consegna dei propri acquisti, ed ogni indirizzo è contrassengato da CAP, via, numero civico, città e nazione di appartenenza. Ogni utente deve anche registrare almeno una carta per effettuare gli acquisti, la quale è definita dal proprio codice identificativo. Infine, il sistema deve poter permettere agli acquirenti di cercare i prodotti della piattaforma per nome, ed ogni acquirente deve poter lasciare un feedback ai prodotti da lui acquistati, rappresentato da un punteggio da 1 a 5 stelle.

    Un prodotto è costituito da un codice identificativo, un nome, una descrizione ed il prezzo. Ogni acquirente può ordinare i vari prodotti presenti all'interno della piattaforma, resi disponibili da fornitori; all'interno di un ordine l'acquirente può scegliere la quantità di ogni prodotto che desidera acquistare, anche di vari fornitori. Il costo dell'acquisto è pari al totale dei prezzi dei prodotti ordinati, sommato al costo di spedizione, il quale è costituito da una tariffa fissa imposta dal servizio di e-commerce, indipendente dai prodotti e dai fornitori. Inoltre, è possibile annullare gli ordini effettuati prima che questi vengano spediti all'acquirente. Se l'acquirente è insoddisfatto di alcuni prodotti ricevuti all'interno di un acquisto, il sistema permette di effettuarne il reso gratuitamente.

    Un fornitore, ovvero un'azienda privata, deve potersi registrare inserendo nel sistema la propria ragione sociale. Inoltre, ogni fornitore deve poter registrare, modificare, eliminare e rifornire i prodotti che desidera mettere in vendita, e può verificare quale siano i prodotti maggiormente venduti in un lasso temporale specificato.

    I trasportatori sono costituiti da aziende private di trasporto merci, che possono registrarsi sulla piattaforma attraverso la propria ragione sociale, ed hanno accesso alla gestione delle spedizioni da loro effettuate; in particolare, si occuperanno di prendere in carico gli ordini a cui non è ancora stato assegnato uno spedizioniere, notificare l'avvenuta consegna di un ordine, l'avvenuto reso o l'eventuale perdita del pacco da consegnare.

    \section{Analisi dei requisiti}

    \begin{enumerate}
        \item Customer
            \begin{enumerate}
                \item registrazione alla piattaforma
                    \begin{enumerate}
                        \item e-mail
                        \item nome
                        \item cognome
                        \item recapito telefonico
                        \item indirizzo
                            \begin{enumerate}
                                \item CAP
                                \item via
                                \item numero civico
                                \item città
                                \item nazione
                            \end{enumerate}
                    \end{enumerate}
                \item registrazione carte
                    \begin{enumerate}
                        \item codice identificativo
                    \end{enumerate}
                \item effettuare ordini di vari prodotti attualmente disponibili
                    \begin{enumerate}
                        \item quantità per ogni prodotto
                        \item carta del customer
                        \item indirizzo di spedizione del customer
                    \end{enumerate}
                \item effettuare resi gratuiti dei propri ordini (dopo l'avvenuta spedizione)
                    \begin{enumerate}
                        \item prodotti e quantità di prodotto da restituire
                    \end{enumerate}
                \item ricerca prodotti per nome
                \item recensire prodotti da loro acquistati attraverso punteggio in [1, 5]
            \end{enumerate}
        \item Fornitore
            \begin{enumerate}
                \item registrazione alla piattaforma
                    \begin{enumerate}
                        \item ragione sociale
                    \end{enumerate}
                \item gestione prodotti (registrazione, modifica, rimozione e rifornimento)
                    \begin{enumerate}
                        \item codice identificativo
                        \item nome
                        \item descrizione
                        \item prezzo
                    \end{enumerate}
                \item statistiche prodotti
                    \begin{enumerate}
                        \item prodotti maggiormente venduti in un dato arco temporale
                    \end{enumerate}
            \end{enumerate}
        \item Trasportatore
            \begin{enumerate}
                \item registrazione alla piattaforma
                    \begin{enumerate}
                        \item ragione sociale
                    \end{enumerate}
                \item gestione spedizioni
                    \begin{enumerate}
                        \item prendere in carico spedizioni ancora non assegnate
                        \item notificare l'avvenuta consegna di ordini
                        \item notificare l'avvenuto reso di ordini
                        \item notificare lo smarrimento di un ordine
                    \end{enumerate}
            \end{enumerate}
    \end{enumerate}

\end{document}
